\documentclass{sigchi}

% Use this command to override the default ACM copyright statement (e.g. for preprints). 
% Consult the conference website for the camera-ready copyright statement.


%% EXAMPLE BEGIN -- HOW TO OVERRIDE THE DEFAULT COPYRIGHT STRIP -- (July 22, 2013 - Paul Baumann)
% \toappear{Permission to make digital or hard copies of all or part of this work for personal or classroom use is 	granted without fee provided that copies are not made or distributed for profit or commercial advantage and that copies bear this notice and the full citation on the first page. Copyrights for components of this work owned by others than ACM must be honored. Abstracting with credit is permitted. To copy otherwise, or republish, to post on servers or to redistribute to lists, requires prior specific permission and/or a fee. Request permissions from permissions@acm.org. \\
% {\emph{CHI'14}}, April 26--May 1, 2014, Toronto, Canada. \\
% Copyright \copyright~2014 ACM ISBN/14/04...\$15.00. \\
% DOI string from ACM form confirmation}
%% EXAMPLE END -- HOW TO OVERRIDE THE DEFAULT COPYRIGHT STRIP -- (July 22, 2013 - Paul Baumann)


% Arabic page numbers for submission. 
% Remove this line to eliminate page numbers for the camera ready copy
\pagenumbering{arabic}

% Load basic packages
\usepackage{tabularx} 
\usepackage{balance}  % to better equalize the last page
\usepackage{graphics} % for EPS, load graphicx instead
\usepackage{times}    % comment if you want LaTeX's default font
\usepackage{url}      % llt: nicely formatted URLs

% llt: Define a global style for URLs, rather that the default one
\makeatletter
\def\url@leostyle{%
  \@ifundefined{selectfont}{\def\UrlFont{\sf}}{\def\UrlFont{\small\bf\ttfamily}}}
\makeatother
\urlstyle{leo}


% To make various LaTeX processors do the right thing with page size.
\def\pprw{8.5in}
\def\pprh{11in}
\special{papersize=\pprw,\pprh}
\setlength{\paperwidth}{\pprw}
\setlength{\paperheight}{\pprh}
\setlength{\pdfpagewidth}{\pprw}
\setlength{\pdfpageheight}{\pprh}

% Make sure hyperref comes last of your loaded packages, 
% to give it a fighting chance of not being over-written, 
% since its job is to redefine many LaTeX commands.
\usepackage[pdftex]{hyperref}
\hypersetup{
pdftitle={SIGCHI Conference Proceedings Format},
pdfauthor={LaTeX},
pdfkeywords={SIGCHI, proceedings, archival format},
bookmarksnumbered,
pdfstartview={FitH},
colorlinks,
citecolor=black,
filecolor=black,
linkcolor=black,
urlcolor=black,
breaklinks=true,
}

% create a shortcut to typeset table headings
\newcommand\tabhead[1]{\small\textbf{#1}}

% shortcut for stakeholder table
\newcommand{\stakeholdertable}{
\begin{table*}
  \centering
  	\begin{tabularx}{0.9\textwidth}{|X|X|X|X|}
    \hline
    \tabhead{Direct Stakeholders} & \tabhead{Benefits/Harms} & \tabhead{Values} & \tabhead{Conflicts} \\
    \hline
    System managers & 
		\underline{Benefits}: Able to fix problems more quickly\newline
		\underline{Benefit or harm}: Less time doing maintenance and tending plants by hand\newline
		\underline{Harm}: Could be alerted of emergencies at any time
	&	Human welfare\newline
	 	Autonomy\newline
		Calmness\newline
		Free time away from work\newline
		Interaction with nature\newline
		Physical interaction with systems\newline
		Awareness (of system functioning)
	& Physical interaction with systems and awareness may conflict with calmness and free time away from work \\
%    \hline
%	Owners of system &
%		\underline{Benefits}: System reduces labor and maintenance costs\newline
%		Produce organic and high quality food\newline
%		\underline{Harms}: Could suffer financial loss if system breaks down
%	&	Ownership and property\newline 
%		Efficiency 
%	& \\
%    \hline
    \hline
    \tabhead{Indirect Stakeholders} & \tabhead{Benefits/Harms} & \tabhead{Values} & \tabhead{Conflicts} \\
    \hline
    Restaurants and\newline restaurant customers &
		\underline{Benefits}: Know about where their food comes from\newline
		Provide feedbacks or improvements to owner\newline
		\underline{Harms}: Could be lied to if presented with false information
	&	Trust\newline
		Accountability\newline
		Environmental sustainability\newline
		Autonomy\newline
		Ownership and property (restaurants)
	&	Ownership and property (in the form of profitability) may compete with environmental sustainability\\
    \hline
  \end{tabularx}
  \caption{Paired down list of stakeholders}
  \label{tab:stakeholders}
\end{table*}}

% End of preamble. Here it comes the document.
\begin{document}

\title{Developing an Aquaponics Interface}

\numberofauthors{3}
\author{
  \alignauthor Justin Bare\\
    \affaddr{University of Washington}\\
    %\email{e-mail address}\\
  \alignauthor Laurel Hart\\
    \affaddr{University of Washington}\\
    \email{hart1a@uw.edu}\\
  \alignauthor Sam Wilson\\
    \affaddr{University of Washington}\\
    %\email{e-mail address}\\
}

\maketitle

\begin{abstract}
Abstract.
\end{abstract}

\keywords{
	Aquaponics; sustainability; value sensitive design;
}

\category{H.5.2.}{Information Interfaces and Presentation}{User Interfaces}

\section{Introduction}

Aquaponics is a method of farming which produces fish and vegetables simultaneously in the same system (see Figure~\ref{fig:skales}). It is particularly useful for increasing food security and producing food sustainably in urban settings. Aquaponics systems are very modular and easy to maintain once they have been set up and equipped with automated monitoring and control devices. With only a few parameters to check every day, even large systems can be managed well by a single person with the right tools. 

The intent of this project is two-fold: to extend the abilities of managers to operate their aquaponics systems effectively and to provide environmentally conscious customers with relevant information about the sustainability of the system. To do this, the project will follow a value sensitive design approach.

\begin{figure*}
\centering
\includegraphics[width=0.9\textwidth]{systemDiagram}
\caption{Skales Cooperative aquaponics system}
\label{fig:skales}
\end{figure*}

\section{Related Work}
\subsubsection{Aquaponics}

As Domingues et. al. demonstrate in \cite{automated}, computer automation of the hydroponic growing process is very effective for increasing efficiency and reducing labor. The current project aims to implement this type of automation for aquaponics systems, which are somewhat similar to hydroponic systems. However, there appears to be a lack of research into the human factors of the design of such an automation system. Based on some initial interviews with experts in the field, urban aquaponics growers have specific needs for novel features of in an interface that would allow them to interact with their systems remotely. The current project has a direct goal of incorporating the values of these experts into the design of an effective online interface for aquaponics systems. 

There is prior work \cite{smallBusiness, cueing} in engaging consumers in thinking about the sustainability of their actions, and also in engaging companies in analyzing the sustainability of their operations \cite{smallBusiness, audit}. Salv\'a et. al. identify several key factors of sustainability that companies and consumers are interested in. Cornelissen et. al. have determined effective ways for engaging people positively in environmental behaviors. Bonanni et. al. have studied a specific tool which allows businesses to present information to customers about the amount of shipping used to produce their products. The current project will combine and expand on these three pieces of work. Specifically, there will be a customer interface showing real time sustainability information (based directly off of the sensor readings) about the operations of the aquaponics system. Customers will be able to see details about water, energy, waste, etc. and how the measures of these sustainability factors compare to the average American farm. The design of the customer interface for a positive experience with environmentally friendly behavior will be well informed by the values of consumers, as determined through surveys, interviews, and other studies. 

In general there seems to be a lack of research into design of well integrated computing technologies for sustainable urban food production and consumption, so this project will be an important step in seeding this area of research in the HCI community. Furthermore, it should demonstrate a helpful addition to the farm-to-table movement by introducing real sustainability information that is available for the consumer to see how their specific actions, in their specific location, benefit the environment, as opposed to generalizing about the environmental benefits of farm-to-table, which can vary significantly from place to place.

\subsubsection{Value Sensitive Design}
Blah~\cite{VSD}\cite{moreVSD}

\section{Methods}

Intro to methods

\subsection{Value Sensitive Design}

Some stuff about VSD

Although we identified an extensive list of potential stakeholders, only a to focus on only a few principle ones in our investigation (see Table~\ref{tab:stakeholders}). Once we identified two key groups, we began by constructing surveys in order to investigate the values of each, which were then adapted into outlines for informal interviews. We had identified some values for each group, but needed to verify whether our 
\stakeholdertable

\subsubsection{Direct Stakeholders}

The direct stakeholders, or system managers, can also be considered the ``expert users'' of our interface, whose needs and skills are specifically adapted to the system at hand. Our informal interviews allowed us to identify a few key parameters that needed to be monitorable using the interface: pH balance, water temperature, nitrogen levels, air temperature, and dissolved oxygen. Ideally the system managers would be able to add any type of parameter that could be monitored, allowing for extensible use. The system managers also stated a preference for all data to be visible on one graph, rather than a separate graph for each parameter. The interviews also confirmed the conflict between valuing calmness and free time away from the system versus physical interaction with and awareness of the system; system managers stated that an online interface would drastically improve maintenance, but that alerts would need to be immediate and disruptive if urgent. 

\begin{figure}[!h]
\centering
\includegraphics[width=0.9\columnwidth]{Sketch1}
\caption{Initial sketch in response to system manager's desire to see all information at a glance.}
\label{fig:sketch1}
\end{figure}

\begin{figure}[!h]
\centering
\includegraphics[width=0.9\columnwidth]{Sketch2}
\caption{Refinement upon first sketch.}
\label{fig:sketch2}
\end{figure}

\begin{figure*}
\centering
\includegraphics[width=0.9\textwidth]{Mockup}
\caption{Color mockup based on D3.js aesthetics.}
\label{fig:mockup}
\end{figure*}

D3.js: \cite{d3js}
 
 
\subsubsection{Indirect Stakeholders}

The group of indirect stakeholders that we addressed, restaurant customers, constitute a much larger group than the direct stakeholders. Although we were only able to conduct interviews with a very small sample of this group, we were still able to elicit some useful information regarding values. 

Initially, the intent of our investigation was to design an interface that addressed the needs of both direct and indirect stakeholders. The intuition was that the information used by system managers to monitor the operations of the aquaponics system could also be used to assess its sustainability. However, our informal interviews made it clear that the information needs of the indirect stakeholders were too different from those of the direct stakeholders to be addressed by the same interface. Instead of implementing another interface, we decided to present suggestions for how such data could be made available to non-experts interested in sustainability:
\begin{itemize}
\item The live data available to system managers is too much for non-experts to make sense of. Instead, certain system information should be available in summary, e.g., water used annually.
\item In addition to presenting information in summary, data should also be contextualized. Even given data such as annual water use, non-experts are not able to compare to more conventional forms of farming. Suggested contextualization techniques include:
	\begin{itemize}
	\item Resources consumed in system to create a plate of food.
	\item Amount of food that could be produced if the aquaponics system filled a city block. 
	\item Direct comparison to conventional farming methods either inhabiting the same amount of space as the aquaponics system or generating the same amount of food.
	\end{itemize}
\item Ideally, information should be available to restaurant customers in-restaurant. Even restaurant customers who self-reported as being particularly interested in sustainability stated that they would not go out of their way (read, download an app or go to a website) to get sustainability information about food served in restaurants. In order to accomodate this, we suggest a system that generates a data sheet suitable for printing, for the restaurants to carry at their discretion. 
\end{itemize}

\section{Results}

Our design efforts resulted in a limited \href{http://homes.cs.washington.edu/~samw11/510/}{live prototype}.\footnote{Accessible for the foreseeable future at \url{http://homes.cs.washington.edu/~samw11/510/}}

\section{Future Work}

\subsubsection{Additional Stakeholders}
Although the original list of stakeholders was thought to be extensive---including everyone who could conceivably be influenced by the aquaponics system or its interface---an interview with Batya Friedman revealed a particular bias: only humans were considered as stakeholders. The fish living in the aquaponics system, possibly the most direct stakeholder of all, had been overlooked. 

Another insight Friedman provided was the consideration of how to handle hardware aging. 

Other extensions to this interface include moving from a passive, monitoring role to being able to actively control certain functions in response to the incoming data, e.g., remotely activating 

\section{Conclusion}

Blah

% Balancing columns in a ref list is a bit of a pain because you
% either use a hack like flushend or balance, or manually insert
% a column break.  http://www.tex.ac.uk/cgi-bin/texfaq2html?label=balance
% multicols doesn't work because we're already in two-column mode,
% and flushend isn't awesome, so I choose balance.  See this
% for more info: http://cs.brown.edu/system/software/latex/doc/balance.pdf
%
% Note that in a perfect world balance wants to be in the first
% column of the last page.
%
% If balance doesn't work for you, you can remove that and
% hard-code a column break into the bbl file right before you
% submit:
%
% http://stackoverflow.com/questions/2149854/how-to-manually-equalize-columns-
% in-an-ieee-paper-if-using-bibtex
%
% Or, just remove \balance and give up on balancing the last page.
%
\balance

\bibliographystyle{acm-sigchi}
\bibliography{FinalProjectReport}
\end{document}
